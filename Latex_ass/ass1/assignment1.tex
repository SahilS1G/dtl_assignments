\documentclass[12pt]{article}
\usepackage{graphicx}
\title{Assignment 1}
\author{Sahil Makeshwar}
\date{November 15, 2022}
\begin{document}
\maketitle
\newpage
\tableofcontents
\newpage
\section{Unit 1}
\subsection{Review of first order differential equations}
\paragraph{}
 An equation that contains derivatives of one or more unknown functions with respect to one or more independent variables is said to be a differential equation. The order of a differential equation matches the order of the highest derivative that appears in the equation.

\subsection{Reduction of order}
\paragraph{}
This method is especially useful for solving second-order homogeneous linear differential equations since (as we will see) it reduces the problem to one of solving relatively simple first- order differential equations.

\subsection{Linear Differential Equations}
\paragraph{}
A linear equation or polynomial, with one or more terms, consisting of the derivatives of the dependent variable with respect to one or more independent variables is known as a linear differential equation.
\newpage

\section{Unit 2}
\subsection{Laplace Transform}
\paragraph{}
Laplace transform is the integral transform of the given derivative function with real variable t to convert into a complex function with variable s.  let f(t) be given and assume the function satisfies certain conditions to be stated later on.

\subsection{Properties}
\paragraph{}
The Laplace transform can also be used to solve differential equations and is used extensively in mechanical engineering and electrical engineering. The Laplace transform reduces a linear differential equation to an algebraic equation, which can then be solved by the formal rules of algebra.

\subsection{Unit step function}
\paragraph{}
The Heaviside step function, or the unit step function, usually denoted by H or 0, is a step function, named after Oliver Heaviside, the value of which is zero for negative arguments and one for positive arguments.
\newpage
\section{Unit 3}
\subsection{Functions of several variables}
\paragraph{}
A function of variables, also called a function of several variables, with domain is a relation that assigns to every ordered -tuple in a unique real number in . We denote this by each of the following types of notation. The range of is the set of all outputs of . It is a subset of , not .

\subsection{Level curves and level surfaces}
\paragraph{}
A level set of a function of two variables f(x,y) is a curve in the two-dimensional xy-plane, called a level curve. A level set of a function of three variables f(x,y,z) is a surface in three-dimensional space, called a level surface.

\subsection{Partial and directional derivatives}
\paragraph{}

\end{document}