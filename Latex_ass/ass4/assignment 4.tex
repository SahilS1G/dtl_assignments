\documentclass[12pt]{article}
\usepackage{natbib}
\usepackage[utf8]{inputenc}
\title{Assignment 4 : Biblogaphy}
\author{Sahil Makeshwar}
\date{November 15, 2022}

\begin{document}
\maketitle
\section{Albert Einstein}
 \subsection{Quotes}
 The following quotes were \cite{Boney96}
 "The most beautiful thing we can experience is the mysterious. It is the source of all true art and all science. He to whom this emotion is a stranger, who can no longer pause to wonder and stand rapt in awe, is as good as dead: his eyes are closed."

\subsection{Book on Einstien}
The book \cite{Boney96} edited by by David E. Rowe and Robert Schulmann is about the the most famous scientist of the twentieth century, Albert Einstein was also one of the century's most outspoken political activists. Deeply engaged with the events of his tumultuous times, from the two world wars and the Holocaust, to the atomic bomb and the Cold War, to the effort to establish a Jewish homeland, Einstein was a remarkably prolific political writer, someone who took courageous and often unpopular stands against nationalism, militarism, anti-Semitism, racism, and McCarthyism. In Einstein on Politics, leading Einstein scholars David Rowe and Robert Schulmann gather Einstein's most important public and private political writings and put them into historical context. The book reveals a little-known Einstein--not the ineffectual and naïve idealist of popular imagination, but a principled, shrewd pragmatist whose stands on political issues reflected the depth of his humanity.

\begin{thebibliography} {}
	
	\bibitem{Boney96} Boney, L., Tewfik, A.H., and Hamdy, K.N., ``Digital
	Watermarks for Audio Signals," \emph{Proceedings of the Third IEEE
		International Conference on Multimedia}, pp. 473-480, June 1996.
	
\end{thebibliography}


\end{document}